\documentclass[11pt]{article}

\usepackage{fancyhdr}
\usepackage{blindtext}
\usepackage{listings}
\usepackage{url}
\usepackage{enumitem}
\usepackage{indentfirst}
\usepackage[parfill]{parskip}
\usepackage{titling}
\usepackage{scrextend}
\usepackage{gensymb}
\usepackage{alltt}
\usepackage{underscore}
\usepackage{cleveref}
\usepackage{xcolor}
\usepackage{fancyvrb}

\topmargin -.5in
\textheight 9in
\oddsidemargin 0in
\evensidemargin 0in
\textwidth 6.5in

\setlength{\droptitle}{-0.25in}

\newenvironment{functionSpec}{
  \begin{addmargin}[0.25in]{0in}
  }
  {
  \end{addmargin}
  }

% ========================================
% Update these lines for each assignment!!
% \newcommand{\hwnum}{GDB Tutorial}
% \newcommand{\hw}{GDB Tutorial}
% \newcommand{\duedate}{}
% ========================================

\lstset{language=C}

% \pagestyle{fancy}
% \fancyhf{}
% \rhead{CS240 - Spring 2019}
% \lhead{\hwnum}
% \rfoot{Page \thepage}

\begin{document}

\title{GDB Tutorial}
\author{Pedro Abreu}
\maketitle

GDB stands for ``GNU Debugger'', and it is a tool that gives you the
power to inspect your program while it is executing. It allows you to
freeze the execution and check the variable values, check if a particular
condition is met, if a function was called, and a lot of other useful
perks that help you to find what is going wrong with your program.

As you may have already noticed, pointers can be quite tricky to
debug. Hopefully this tutorial will make your life a lot easier from
here on.

To follow this tutorial clone the {\tt gdb-tutorial} directory just like any
other homework
\begin{verbatim}
$ git clone https://github.com/pedrotst/gdb-tutorial
\end{verbatim}

Navigate to the {\tt src} folder and you should see the following {\tt main.c}:
\begin{verbatim}
#include <stdlib.h> 
#include <stdio.h> 
#include "my_list.h" 
 
int main(){ 
    my_list *l = NULL; 
 
    l = insert_node(1, l); 
    l = insert_node(2, l); 
    l = insert_node(3, l); 
    l = insert_node(4, l); 
 
    print_list(l); 
    delete_list(l); 
 
    return 0; 
}
\end{verbatim}

This program uses the provided {\tt my\_list} implementation to create
a singly-linked list. This list should have four elements. Let's try to
run this code and see what happens...

\clearpage

Start by compiling {\tt my\_list.c} to an object file:
\begin{verbatim}
$ gcc -Wall -Werror -std=c99 -c my_list.c
\end{verbatim}

Now compile and run the {\tt main} executable:
\begin{verbatim}
$ gcc -Wall -Werror -std=c99 main.c my_list.o -o main
$ ./main
Segmentation fault
\end{verbatim}

Uh-oh, {\tt Segmentation fault}! And conveniently, it doesn't even tell us
where the program failed. There are a number of things we can do at this
point. We could sprinkle {\tt printf()}s and {\tt fflush()}s all over, for
instance. Another solution, unsuprisingly at this point, is GDB!

In order to use GDB, we must first take a step back and recompile our code
with the {\tt -g} flag. This creates an object file that contains ``debug
symbols''. These symbols are leveraged by GDB to provide more detailed and
useful information to the programmer.

\begin{verbatim}
$ gcc -Wall -Werror -std=c99 -g -c my_list.c
$ gcc -Wall -Werror -std=c99 -g main.c my_list.o -o main
\end{verbatim}

\textit{Note: when you are implementing the homeworks, you don't have
to worry about the {\tt -g} flag because it is already included in the
{\tt Makefile}. It is worthwhile, however, to rememeber this for your
future endeavors.}

With the files compiled with debug symbols, we have everything we need
to use GDB:

\begin{verbatim}
$ gdb main
GNU gdb (Ubuntu 8.1-0ubuntu3) 8.1.0.20180409-git
Copyright (C) 2018 Free Software Foundation, Inc.
License GPLv3+: GNU GPL version 3 or later <http://gnu.org/licenses/gpl.html>
This is free software: you are free to change and redistribute it.
There is NO WARRANTY, to the extent permitted by law.  Type "show copying"
and "show warranty" for details.
This GDB was configured as "x86_64-linux-gnu".
Type "show configuration" for configuration details.
For bug reporting instructions, please see:
<http://www.gnu.org/software/gdb/bugs/>.
Find the GDB manual and other documentation resources online at:
<http://www.gnu.org/software/gdb/documentation/>.
For help, type "help".
Type "apropos word" to search for commands related to "word"...
Reading symbols from main...done.
(gdb) 
\end{verbatim}

Alright! {\tt Reading symbols from main...done.} means that GDB was able
to successfully read the debug symbols that we included in the previous
step.

\clearpage

We can now run the program...

\begin{Verbatim}[commandchars=\\\{\}]
(gdb) run
Starting program: /u/antor/u7/\$USER/cs240-hws/gdb-tutorial/example/main 
Program received signal SIGSEGV, Segmentation fault.
{\color{red}0x00005555555548f4 in insert_node (val=2, l=0x555555756260) at my_list.c:31}
31	   last->next = create_node(val);
\end{Verbatim}

This is quite useful! GDB is able to provide us with the exact location
in the our code where the crash occurred. We can see this is on line 31
inside {\tt my\_list.c}. This is part of the {\tt insert\_node()} function.
 line marked in red means that the program crashed

Let's set a breakpoint on this line. This will cause the program to hault
\emph{before} executing the code on line 31...

\begin{verbatim}
(gdb) break my_list.c:31
Breakpoint 1 at 0x8e3: file my_list.c, line 31.
\end{verbatim}

Now we run the program again...

\begin{verbatim}
(gdb) r
The program being debugged has been started already.
Start it from the beginning? (y or n) y
Starting program: /u/antor/u7/pdacost/cs240-hws/gdb-tutorial/example/main 

Breakpoint 1, insert_node (val=2, l=0x555555756260) at my_list.c:31
31	   last->next = create_node(val);
\end{verbatim}

\textit{Did you notice that we used {\tt r} instead of {\tt run}? This is
  because gdb allows us to be lazy and type only the first letter or two
  of a command.  This means we can also use {\tt b}, {\tt p},  {\tt w},
  for break, print and watch respectively!}

Some notes on breakpoints. First, it is possible to set multiple breakpoints.
You can also break on function names instead of line numbers. To view
the current list of breakpoints, you can use the following command:

\begin{verbatim}
(gdb) i b
Num     Type           Disp Enb Address            What
1       breakpoint     keep y   0x0000000000401299 in insert_node 
                                                   at my_list.c:31
\end{verbatim}

This is short for ``info breakpoints''. To remove a breakpoint, we can
run {\tt d 1} where 1 is the number. ``d'' is short for ``delete.''

Let's take a look on the code surrounding line.

\begin{verbatim}
(gdb) wh
\end{verbatim}

\textit{You can also do this directly on launch, by running {\tt gdb
-tui main} instead.}

This command should cause gdb to display the code surrounding the line
in question. To close this view, press \textit{Ctrl-C a}.

\clearpage

Alternatively, you can also use the {\tt list} command:
\begin{verbatim}
(gdb) list
\end{verbatim}

We can also now inspect the value of the variables:

\begin{verbatim}
(gdb) print last
$1 = (my_list *) 0x0
\end{verbatim}

This means {\tt last} has type {\tt my\_list *} and has value {\tt 0x0},
which of course is NULL. So when we run line 31 it will try to dereference
a NULL pointer! This is the cause of our crash.

Let's execute the next line anyway:

\begin{verbatim}
(gdb) next
Program terminated with signal SIGSEGV, Segmentation fault.
\end{verbatim}

Take a minute to open {\tt my\_list.c} and fix the bug.  Once fixed,
you should should see the following output:

\begin{verbatim}
$ main
[1, 2, 3, 4]
\end{verbatim}

\textit{Don't forget to recompile your code when you are done fixing
the issue(s)!}

Already with just these few commands, you will be able to debug many
programs.

To wrap up, here is a list of some other commands you may find useful
in the future:

\begin{itemize}[noitemsep]
  \item {\tt step}: Execute next program line (after stopping); step into any
function calls in the line;
  \item {\tt watch}: Every time a given variable changes it's value the program stops;
  \item {\tt continue}: Continue running your program (after stopping, e.g. at a breakpoint);
  \item {\tt clear}: Clears breakpoints;
  \item {\tt backtrace}:  Shows the backtrace of your program;
  \item {\tt wh}:  Shows a window with the surrounding code being executed.
\end{itemize}

For a list of all the commands run
\begin{verbatim}
(gdb) help
\end{verbatim}

\textit{Remember: The best way to learn how to program is by sitting down and
programming yourself! There is no magic.}

\end{document}
